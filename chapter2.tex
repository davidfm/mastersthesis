

\chapter{The impact of the weather data}

Even if it is obvious for all of us, weather is one of the most important factors of the environment, with a high impact in our life. At the same time most of us are not familiar with the repercussions of the weather, what is causing different phenomena and the implications of them. Finally, our needs concerning the weather are limited by the availability of the data that is given to us. The role of the weather forecast broadcasting resides in different organizations. However, the advantages of the technology are bringing us the capability to have a more frequent and reliable access. The following sections analyzes how the weather data is spread and in which points of its diffusion can be improved.

\section{Weather data collection and diffusion}\label{2.1}

Depending of the region of the world, we can find more or less geographical locations in which a weather station has been placed to collect information about different phenomena. It is important to clarify they are several categories of phenomena with different needs in terms of data collection requirements. In addition, we have different units and time frequencies to make this data useful.

Fortunately, nowadays, most of the known phenomena have a solid basement of understanding, meaning this that we can measure them and get some conclusions and to act in consequence.  The \gls{SI} is used as the recognized standard of units for these measurements\footnote{Some countries as Burma, Liberia, the United States or the United Kingdom, have other local standards coexisting with the \gls{SI}. This implies some adaptions concerning the weather data. Due to the local units it is necessary to include unit conversions in the data manipulation process.}.

The figure \ref{f2.1} shows the scenario abstracting the data to a generic input:

\begin{figure}[H]
\centerline{\includegraphics[width=1\textwidth]{images/c1f3.png}}
\caption{Layers abstracted in the weather collection data workflow.}
\label{f2.1}
\end{figure}

As we can see the scenario gives as an abstract input of data from the different environmental phenomena. After that, the data is sent to the data processing center (commonly a governmental \& scientist organizations). At the end, the data is interpreted and the conclusions are spread. The Physics are giving us the possibility to understand these phenomena based in the observation and correlation of them; for this it is needed to establish direct dependencies between the phenomena.

\begin{figure}[H]
\centerline{\includegraphics[width=1\textwidth]{images/c1f4.png}}
\caption{Weather data collection workflow. World Climate Data and Monitoring Programme.\protect\footnotemark}
\end{figure}

Commonly, we can find several governmental and scientist organizations around the world, focused in the weather data collection. As example of this,in Finland we have the Finnish Meteorological Institute(FMI) \cite{FMI}, or different example can be a worldwide organizations such as the \gls{WMO}\cite{WMO}, in charge of the coordination of the exchange and collection of weather data between organizations around the world. These organizations are the official source of information for weather data. Even so, they are not the only ones.

\footnotetext{The World Climate Data and Monitoring Programme (WCDMP) is a programme of the World Climate Programme that facilitates the effective collection and management of climate data and the monitoring of the global climate system, including the detection and assessment of climate variability and changes.\cite{WMO}}

Thousands of individuals are helping with the weather data collection as well. Those individuals in possession of some weather instruments can collaborate transmitting the data to some governmental organization, for instance the program \gls{CWOP} \cite{CWOP} has over \textbf{20,000 members in 149 countries}. This is possible using technologies like \gls{APRS} \cite{APRS} system, which is mentioned in CWOP website\cite{CWOP} as the following: 

\emph{"The Automatic Position Reporting System (APRS) is a part of ham radio that provides an ideal way for weather station operators to distribute their weather data much further than the regions within their transmitter range. APRS was originally intended for position information data but actually provides a means for automatic transmission of all sorts of digital data. This is especially true now that the original APRS packet radio concept has been enhanced to include the capabilities of the Internet. The reporting of citizen weather data is a particularly useful application of the APRS Internet Service (APRS-IS)."} 

\subsection{Governmental organizations}

Denominated as meteorological institutes or meteorological agencies, it is possible to find a big group of organizations around the world, which purpose is to study the weather. Almost all of these organizations are funded by the governments, moreover of these state and local organizations, other country-region organizations exist to coordinate the study of the weather in a bigger extension area. As an example, the \gls{FMI}\cite{FMI} is in charge of studying the weather in the region of Finland. At the same time the \gls{FMI} is member of the \gls{ECMWF}\cite{ECMWF}, organization in charge \emph{"to provide operational medium- and extended-range forecasts and a state-of-the-art super-computing facility for scientific research." }.The same scenario can be found in different continents as America with organizations as \gls{NOAA}\cite{NOAA}.

These worldwide organizations are creating the infrastructure to collect the weather data around the world. It is necessary to highlight that the study of the weather is an expensive activity, involving a big amount of resources such as high-tech instruments, installation of these instruments in different locations (with the extra cost that it implies) and use of computation centers to evaluate the data. Due to these facts, we can find that the amount of weather stations around the world and the effort or size of these organizations can vary significantly depending of the economy of the region. This means that the weather infrastructure in the occidental world is well designed, implemented and functional. However, in other areas like Africa, the amount of available weather stations decrease for economical reasons. In addition, and due to the nature of the weather, organizations like \gls{NOAA} and \gls{ECMWF} are installing weather collection points outside their official operation areas\footnote{Both organizations are restricted to America and Europe, nevertheless, these organizations have permission to place collection points out of their area to improve the quality of the studies and to encourage the international cooperation.}, thus getting better samples to evaluate the global weather conditions.

These state-region organizations have a huge cooperation between them. Scientists are pretty conscious about the need to get samples of weather data from different regions to evaluate it, thus, they are fomenting the cooperation of the weather data exchange. The \gls{WMO} defined the proceedings of measurement for meteorological variables\cite{GMIMO}, providing a common basement to perform the measurements related with the weather. Furthermore, the \gls{WMO} is conscious about the issue of data exchange, in chapter four the process of standardization that \gls{WMO} is supporting and the issues of it are analyzed deeply.

\subsection{Corporations}

As it was mentioned previously, weather has a big impact in our life. It implies that not only practical advantages can be extracted from the study of it, also the study of the weather is generating a big range of economical activities. 

Industries like construction or military, have even more interest in know which phenomena are occurring and the future predictions of them. This interest have fomented a whole parallel industry of services of weather data reports.

At the same time, some professional forecast services have appeared as an alternative for independent studies in particular regions of the world. Although this economical activity is mainly deployed by private corporations some governmental organizations are offering also private services.

\subsection{Individuals}\label{2.1.3}

The program \gls{CWOP} mentioned in the section \ref{2.1}, is a perfect example about how individuals can help to collect and to study the weather data. Furthermore, non official programmes have been appearing around the world; using the Internet as foundation, different communities of weather observers are contributing to create individual networks of data exchange, in which a user can access the data of different weather stations around the world. 
\begin{figure}[H]
\centerline{\includegraphics[width=0.7\textwidth]{images/c1f5.png}}
\caption{Meteoclimat screenshot showing weather forecasts.\protect\footnotemark}
\end{figure}
\footnotetext{This data is collected by individuals that have installed a specify software in their computers to send the data to Meteoclimatic servers.}

Meteoclimatic\cite{METEOCLIMATIC}, is a good example of this:" a big network of automatic non professional weather stations", in which hundreds of users share the data collected from their weather stations without any commercial purpose. 
Often, these communities share efforts with governmental organizations in programs as \gls{CWOP}, however, the turn up of theses communities are supported by the demand of the users to have a system in which their data is useful for other individuals, and at the same time give them some independency from governmental  organizations, in terms of data availability.


\subsection{Weather data publication}

The previous sections mention which organizations are involved on the process of data collection. However, the process does not end here; after the collection and evaluation of the data, the final step is to spread and make it useful. The implications of the broadcasting concerning the weather forecast are multiple and they are out of the scope of this thesis. Even so, the spreading of the data is limited for the protocols used in the acquisition of it. As mentioned in section \ref{2.1.3}, some communities of individuals appeared, taking the role of data availability disposal for the end user. Proving this the fact that the way in which the information is managed by the governmental and private organizations, sometimes does not fit with the end user's wishes.

In the past, the weather forecast was delivered through traditional methods as newspapers, radio and TV. Nevertheless, nowadays the Internet has taken this role in several aspects. Almost, all the governmental weather organizations mentioned in this chapter have a web site in which they publish -in different quantities and formats-, the information collected and extracted from their meteorological networks. Although traditional media still report the daily forecast, the tendency points to the Internet as the future mainstream channel of this information.

In addition, other commercial web sites offer this information partially free of charge. This practice caused the appearance of several sites offering \gls{API} services to fetch weather data, giving the possibility to the developers to get some storage data to perform some operations. Due to this \gls{API} availability, some organizations non related directly with the weather data collection workflow, have published some web sites that are exposing data fetched from different APIs and providing a different range of alternatives to the users.

\begin{figure}[H]
\centerline{\includegraphics[width=0.7\textwidth]{images/c1f6.png}}
\caption{\protect \gls{FMI} website \protect \cite{FMI} spreading local weather observations.}
\end{figure}

The author could not find any \gls{API} offering the capability to connect directly to the weather instruments to fetch RAW data streams; all the APIs available are offering pre-processed data. 

\section{Summary}

In this chapter we have given general background information in order to make the scope of the thesis more familiar, in terms of which organizations are in charge of the weather collection and the structure and collaboration between them. Also, it has been analyzed how different organizations of the same field coexist.

We discussed how the same activity is performed in different layers, being involved in the process from official organizations to individuals. Some schemas have been presented, giving a global vision about how the weather data workflow works.

We know now that there is even a global organization named \gls{WMO}. This organization is only dictating some guidelines to perform the measurements. The next chapter introduces a general overview of a weather instrument, to understand how it works, its technologies and limitations.

In the next chapter some concepts and scenarios are explained to understand how a weather instrument works, the technologies that are conforming it, and giving us a global vision of the technologies to have in consideration when we are implementing a protocol for a weather instrument.

\pagebreak
