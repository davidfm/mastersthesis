\chapter{Conclussions}

This thesis has presented the increased popularity of small satellites have been presented, examples of this are Aalto-1 and ESTCube-1, and the difficulties found when trying to find suitable software for a ground station. From the very beginning, each mission has created their own applications, with no cooperation between them.\\

Understanding satellite-Earth communications is crucial when designing a solution for a ground station. As the satellite is moving at high speed around Earth, orbits affect these communications. It is also important to understand how they are represented on a map and how their information is delivered so they can be processed by computers. In addition, the different types of data exchanged between the satellite and the ground station have been covered, including the beacon, and telemetry, where its values are usually useless for the scientists, so they need to be calibrated. The telemetry also includes the housekeeping data, which values needs to be checked against limits to assert the well being of the satellite. To complete the study of satellite-Earth communications the different parts of a ground station have been presented, including hardware and software. Finally, the most common protocols for amateur satellite communications where listed.\\

\emph{Hummingbird} is an open source project aiming to create software which can be easily adaptable for any kind of mission as well as creating a network of ground stations. It is being developed by CGI in cooperation with the University of Tartu and some external collaborators, such as the author of this thesis.\\ 

As part of this collaboration two \emph{Hummingbird} modules have been designed and implemented. The first one being the telemetry calibration module and the second one the limit checker. 

The most challenging parts of this work have been related to the calibration module. As an algorithm which adapted to the way modules communicate in \emph{Hummingbird} had to be design. The other challenge was finding a way to allow scientists to input calibration information without the need of changing the Java code; the solution chosen was to use an embedded  interpreter, so anyone with knowledge of scripting would be able to create their own calibration scripts.\\

The result has been two fully configurable and adaptable modules, which allow to calibrate and check the limits of the parameters received from the satellite. If there are changes in the mission, only the configuration files need to be changed, making this much easier than recoding and recompiling.\\