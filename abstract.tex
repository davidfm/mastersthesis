%ABSTRACT
\thispagestyle{empty}
\begin{tabular}{l r}
\begin{minipage}[c]{102mm} 
\textsc{Aalto University}\\ \textsc{School of Electrical Engineering}
\end{minipage}
 & \hfill
\begin{minipage}[c]{40mm} 
\begin{flushright}
\textsc{Abstract of the}\\ \textsc{Final Project}
\end{flushright}
\end{minipage}\\ 
\end{tabular}

\begin{tabular}{|l|}
\hline %
\begin{minipage}[c][26mm]{\textwidth} 
\vfill
\textbf{Author:} David Fernández \vfill
\textbf{Title:} Development of calibration and limit checking modules for a satellite's ground control software \vfill
\textbf{Date:} 31.08.2013 \hfill \textbf{Language:} English \hfill \textbf{Number of pages:} 69 \vfill
\vfill
\end{minipage}\\ 
\hline

\begin{minipage}[c][18mm]{\textwidth} 
\vfill
School of Electrical Engineering \vfill
Deparment of Radio Science and Engineering \vfill
\textbf{Professorship:} Space Technology \hfill \textbf{Code:} S-92\vfill
\vfill
\end{minipage}\\
\hline

\begin{minipage}[c][12mm]{\textwidth} 
\vfill
\textbf{Supervisor:} D.Sc. (Tech.) Jaan Praks \vfill
\vfill
\end{minipage}\\
\hline
\begin{minipage}[t][133mm]{\textwidth} %Real Abstract
The goal of this project is to develop a calibration and a limit checking modules for Hummingbird, an open source software platform for controlling ground stations and satellites which being developed by CGI Group Inc. in cooperation with the University of Tartu. The goal of Hummingbird is to provide satellite missions with and software easy adaptable to every mission needs along with creating a ground stations network to wide their transmission capabilities.\\

To give the reader some background in satellite communications several parts which need to be taken into account are presented. Orbits and how they affect those transmissions, which data is transmitted, what protocols are used for those communications and the different parts of hardware needed for a ground station to work. In addition, some examples of software used for ground stations are listed.\\

As the main goal of this work is the integration of its two resulting modules with Hummingbird, the platform is presented. Specially focusing on its architecture, which technologies it uses and the functionalities it offers.\\ 

The software design made in this work was developed in close cooperation with the scientists working on the Estonian student satellite. The result is a completely configurable software prepared to be fully integrated with Hummingbird when necessary.
\vfill

\end{minipage}\\
\hline

\begin{minipage}[c][12mm]{\textwidth}
\vfill
\textbf{Keywords:} nanosatellite, ground station, calibration, limits, communications, orbits, Hummingbird
\vfill
\end{minipage}\\
\hline
\end{tabular}

\pagebreak